\documentclass{beamer}
\usepackage[utf8]{inputenc}
\usetheme{Warsaw}

\title{Présentation CMMAI}
\author{H4203} %TODO

\begin{document}

\begin{frame}
\titlepage
\end{frame}

\begin{frame}
Introduction
\end{frame}

\begin{frame}
éventuelles spécifications venant compléter le sujet
TODO
\end{frame}

\begin{frame}
LSG simplifé de l'application globale avec le découpage en 3 lots fonctionnels
Schema LSG TODO
\end{frame}

\begin{frame}
IHM du poste distant
connection
\end{frame}

\begin{frame}
IHM du poste distant
configuration
\end{frame}

\begin{frame}
IHM du poste distant
log
\end{frame}

\begin{frame}
IHM du poste distant
erreur \/ warning
\end{frame}

\begin{frame}
\frametitle{Lot 1 : réseau, journalisation, gestion des évènements.}
    \begin{block}{Réalisé par Paul et Maxime}
	\begin{itemize}
	    \item   Communication réseau en entrée et en sortie du système.
	    \item   Communication des évènements par boite aux lettres à
	    l'intérieur du système.
	    \item   Gestion de la journalisation sur disque.
	\end{itemize}
    \end{block}
\end{frame}

\begin{frame}

\begin{frame}
    \frametitle{LCG : Réseau, en entrée}
     
\end{frame}

% Pas exactement de choix pour notre partie : on communique juste nos choix en
% ce qui concerne le protocole.
\begin{frame}
\frametitle{Protocole}
    \begin{block}{Choix}
	\begin{itemize}
	    \item Protocole \texttt{plain text}.
	    \item Séparateur : retour chariot.
	\end{itemize}
    \end{block}
\end{frame}

\begin{frame}
Binôme 2
- LCG détaillé et complet en mode simulation
\end{frame}

\begin{frame}
Binôme 2
- justification des choix effectués pour la simulation  
\end{frame}

\begin{frame}
Binôme 3
- LCG détaillé et complet en mode simulation
\end{frame}

\begin{frame}
Binôme 3
- justification des choix effectués pour la simulation 
\end{frame}

\begin{frame}
- Intégration (démarche, plan, tests, resultats)
%TODO
\end{frame}

\begin{frame}
- démonstration de vos réalisations
démonstration externe au slides ?
\end{frame}

\begin{frame}
- bilan du projet (auto-critique, améliorations possibles, points forts/faibles, difficultés rencontrées ...) 
%TODO
\end{frame}
\end{document}

