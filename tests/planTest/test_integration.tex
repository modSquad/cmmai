\section{Général}

\Test{Général}{Ordre de lancement}
{Un ordre de lancement est envoyé au système sans qu'aucune configuration n'ai précédemment été envoyé}
{L'ordre doit être ignoré et le système rester dans un état d'attente.}

\section{Rebut}
\Test{Rebut}{Visualiser le taux de pièces acceptées}
{Lancer le programme, et envoyer une configuration depuis le client Windows, envoyer un ordre de lancement}
{À chaque carton traité, le taux de pièces acceptées doit être mis à jour dans l'interface de l'application Windows.}
TODO : AJOUTER au LOG

\Test{Rebut}{Rebuter une pièce si elle est défectueuse}
{Lancer le programme, envoyer une configuration depuis le client Windows et envoyer l'ordre de lancement}
{Lorsqu'une pièce défectueuse est détectée le clapet de rebut doit s'ouvrir, rejeter la pièce et ajouter une entrée dans le log}
TODO : AJOUTER au LOG

\Test{Rebut}{Rebuter une pièce si elle est défectueuse}
{Lancer le programme, envoyer une configuration depuis le client Windows et envoyer l'ordre de lancement}
{Lorsqu'une pièce correcte est détectée le clapet de rebut doit rester fermé, accepter la pièce et ajouter une entrée dans le log}
TODO : AJOUTER au LOG

\Test{Rebut}{Seuil maximal de pièce défectueuse dépassé}
{Lancer le programme, et envoyer une configuration depuis le client Windows, envoyer un ordre de lancement
et attendre que le seuil soit dépassé.}
{Le clapet doit être fermée et une erreur lancée.}

\section{Impression}
\Test{Impression}{Cas nominal}
{Un carton est rempli.}
{L'impression d'une étiquette contenant toutes les information du carton doit être réalisée par l'une des deux imprimante}

\Test{Impression}{Imprimante en panne}
{Le système détecte qu'une imprimante est tombée en panne.}
{une anomalie est générée et le voyant d'anomalie est allumé.}

\Test{Impression}{Une seule imprimante en panne}
{Un carton est rempli, une des deux imprimante est en panne.}
{L'impression d'une étiquette contenant toutes les informations du carton doit être réalisée par l'imprimante fonctionnelle. }

\Test{Impression}{File pleine}
{La file d'attente de carton en attente d'impression est pleine (5 cartons dans la file).}
{Une erreur est générée, et le clapet d'arrivé est fermé.}

\section{Journalisation}
\Test{Journalisation}{Cas nominal}
{Un évènement générant une entrée dans le fichier de journalisation à été lancé}
{Le journal doit être lisible sur le disque et contenir :
\begin{itemize}
	\item La date d'ajout au fichier 
	\item L'heure d'ajout au fichier
	\item Les caractéristiques du lot : Nombre de pièces, le nombre de pièces défectueuses, code défaut, code opération, code opérateur
\end{itemize}
}

\Test{Journalisation}{Cas nominal}
{Toute erreur ou anomalie doit être signalée dans les journaux.}
{Lancer l'application dans un fonctionnement pouvant générer des erreurs et des anomalies}
{Le journal doit contenir la trace de chacune des erreurs et des anomalies survenues.}

\section{Arrêt d'urgence}
\Test{Arrêt d'urgence}{Cas nominal}
{}
{}
