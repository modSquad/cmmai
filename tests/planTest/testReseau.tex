\Test{Réseau}{Recevoir une configuration depuis le client}
{Lancer le programme, et envoyer une configuration depuis le client Windows.}
{La configuration doit être reçue.}

\Test{Réseau}{Recevoir un ordre de lancement}
{Lancer le programme, envoyer une configuration, puis envoyer un ordre de
lancement depuis le client Windows.}
{La configuration doit être reçu, et le système industriel doit se mettre en
fonctionnement, selon la configuration reçue.}

\Test{Réseau}{Dire à l'application de s'arrêter}
{Lancer le programme, envoyer une configuration, envoyer un ordre de lancement,
puis patienter jusqu'à la fin, et enfin, envoyer un ordre d'arrêt, juste
avant le dernier carton.}
{Le programme doit se fermer à la fin du dernier carton.}

\Test{Réseau}{Tester la reprise sur erreur}
{Lancer le programme, envoyer une configuration, envoyer un ordre de lancement,
et appuyer sur le bouton d'arrêt d'urgence. Envoyer ensuite une commande de
    reprise depuis le client Windows.}
{Le système industriel doit se remettre à fonctionner.}

\Test{Réseau}{Envoyer une chaine non valide}
{Lancer le programme, puis envoyer une chaine non valide par le réseau, à
    l'aider d'un utilitaire de type \kw{netcat}.}
{Le système industriel doit ignorer la commande.}

\Test{Réseau}{Envoyer un ordre de configuration avec des valeur non valides}
{Lancer le programme, envoyer une configuration non valide depuis un utilitaire
    de type \kw{netcat}.}
{Le système industriel doit ignorer la commande.}
