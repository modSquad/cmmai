\Test{Réseau~/~réception}{Recevoir une configuration depuis le client}
{Lancer le programme, et envoyer une configuration depuis le client Windows.}
{La configuration doit être reçue.}

\Test{Réseau~/~réception}{Recevoir un ordre de lancement}
{Lancer le programme, envoyer une configuration, puis envoyer un ordre de
lancement depuis le client Windows.}
{La configuration doit être reçu, et le système industriel doit se mettre en
fonctionnement, selon la configuration reçue.}

\Test{Réseau~/~réception}{Dire à l'application de s'arrêter}
{Lancer le programme, envoyer une configuration, envoyer un ordre de lancement,
puis patienter jusqu'à la fin, et enfin, envoyer un ordre d'arrêt, juste
avant le dernier carton.}
{Le programme doit se fermer à la fin du dernier carton.}

\Test{Réseau~/~réception}{Tester la reprise sur erreur}
{Lancer le programme, envoyer une configuration, envoyer un ordre de lancement,
et appuyer sur le bouton d'arrêt d'urgence. Envoyer ensuite une commande de
    reprise depuis le client Windows.}
{Le système industriel doit se remettre à fonctionner.}

\Test{Réseau~/~réception}{Message de reprise sur erreur alors que le système n'est pas
    arrêté}
{Lancer le programme, envoyer une configuration, envoyer un ordre de lancement.
Envoyer ensuite un ordre de reprise sur erreur alors que le programme n'est pas
en erreur.}
{Le programme doit ignorer la commande.}

\Test{Réseau~/~réception}{Envoyer une chaine non valide}
{Lancer le programme, puis envoyer une chaine non valide par le réseau, à
    l'aider d'un utilitaire de type \kw{netcat}.}
{Le système industriel doit ignorer la commande.}

\Test{Réseau~/~réception}{Envoyer un ordre de configuration avec des valeur non valides}
{Lancer le programme, envoyer une configuration non valide depuis un utilitaire
    de type \kw{netcat}.}
{Le système industriel doit ignorer la commande.}

\Test{Réseau~/~émission}{Envoyer un message d'acceptation de pièce}
{Lancer le programme, envoyer une configuration valide, et lancer le système.}
{Pour chaque pièce sensée être acceptée par le système, un message doit être
    envoyé par le système. Le message doit respecter le protocole.}

\Test{Réseau~/~émission}{Envoyer un message comme quoi une pièce a été rejeté
    du système}
{Lancer le programme, envoyer une configuration valide, et lancer le système}
{Pour chaque pièce sensée être rejetée par le système, un message doit être
envoyé par le système. Le message doit respecter le protocole.}

\Test{Réseau~/~émission}{Envoyer un message si trop de pièces sont
    défectueuses}
{Lancer le système, et attendre qu'il y ait trop de pièce défectueuse.}
{Quand trop de pièces sont défectueuses, le système doit envoyer un message. Ce
message doit respecter le protocole.}

\Test{Réseau~/~émission}{Envoyer un message si la file d'attente pour
    l'impression est vide}
{Lancer le système, et attendre qu'il y ait trop de cartons dans la file
    d'attente.}
{Quand trop de cartons sont dans le file d'attente, le système doit envoyer un
    message. Ce message doit respecter le protocole.}

\Test{Réseau~/~émission}{Envoyer un message lors d'une famine de pièce}
{Lancer le système, et attendre une famine de pièce (pas de pièce qui sort
depuis $n$ secondes.}
{Un message doit être envoyé, précisant qu'il y a famine. Ce message doit
respecter le protocole.}

\Test{Réseau~/~émission}{Envoyer un message lors d'une famine de cartons}
{Lancer le système, et attendre une famine de cartons (pas de cartons alors
qu'un carton est demandé).}
{Un message doit être envoyé, précisant qu'il y a famine. Ce message doit
respecter le protocole.}

\Test{Réseau~/~émission}{Envoyer un message lorsqu'une imprimante est en
panne}
{Lancer le système, et attendre qu'une imprimante tombe en panne.}
{La première fois qu'une imprimante est en panne et doit imprimer, un message
doit être envoyé, précisant quelle imprimante est en panne. Ce message doit
être conforme au protocole.}

\Test{Réseau~/~émission}{Envoyer un message de journalisation}
{Lancer le système, et attendre que quelques évènements surviennent.}
{Pour chaque évènement, un message doit être envoyé. Ce message doit respecter
le protocole.}
