% kw, pour keyword : permet de mettre en chasse fixe, pour par exemple
% écrire le nom d'une fonction ou d'un nom de variable, etc.
\newcommand{\kw}[1]{\texttt{#1}}

% Fonte qui sera utilisé pour le titre d'un test
\newcommand{\fontptitle}[1]{%
    {\huge{\texttt{\textbf{#1}}}}
}

% Largeur du titre (le texte)
\newlength{\largeurtitre}
% Largeur de la ligne (
\newlength{\largeurligne}
% Permet de faire une ligne à droite d'un texte, pour effectuer une
% séparation visuelle plus nette.
\newcommand{\lignedroite}[1]{%
    \setlength{\largeurtitre}{0pt}
    \setlength{\largeurligne}{0pt}
    \settowidth{\largeurtitre}{\fontptitle{#1}}
    \addtolength{\largeurligne}{\textwidth}
    \addtolength{\largeurligne}{-\largeurtitre}
    % On ajoute quand même une petite séparation.
    \addtolength{\largeurligne}{-4pt}
    \bigskip
    #1 \raisebox{0.3em}{\vrule depth 0pt height 0.2ex width \largeurligne}
}

% Commande pour écrire un test
% $1 : Nom du module
% $2 : Nom du test
% $3 : Description du test
% $4 : Résultats attendus
\newcommand{\Test}[4]
{%
	\addtocounter{testno}{1}
	\begin{minipage}{\linewidth}
	\lignedroite{\textbf{\large{Module #1 -- \kw{Test \thetestno}~-- #2}}}\\
	\textbf{\large{Description}}\\
	#3 \\
	\textbf{\large{Résultat attendu}}\\
	#4 
	\end{minipage}
	\vspace{1cm}
}

