\Test{Client Windows}
    {Connection à un serveur}
    {Entrer l'adresse et le port du serveur, cliquer sur le bouton "connecter".}
    {Le client doit se connecter.
    Le bouton connecter laisse place à un bouton déconnecter.
    Les différents widgets se dégrisent pour permettre d'agir sur le server.\\
    Remarque : Si une mauvaise addresse, ou une adresse suivant un mauvait format est rentré, le comportement est incertain.}

Les tests suivant impliquent que le client soit connecté à un serveur.

\Test{Client Windows}
    {Déconnection d'un server}
    {Cliquer sur le bouton "deconnecter".}
    {Le client doit se déconnecter.
    Le bouton déconnecter laisse place à un bouton connecter.
    Les différents widgets se grisent empechant l'utilisateur d'agir hors connection.}

\Test{Client Windows}
{Reception d'un message de log}
    {A l'aide d'un serveur, ou d'un socket serveur, envoyer un message de log.}
    {Le message doit s'inscrire dans la liste des messages de log.}

\Test{Client Windows}
{Reception d'un message d'acceptation de pièces}
    {A l'aide d'un serveur, ou d'un socket serveur, envoyer un message d'acceptation de pièces.}
    {L'indication du nombre de pièces acceptés indiqué doit s'incrémenter de la valeur envoyes dans le méssage.}

\Test{Client Windows}
{Reception d'un message de rejet de pièces}
    {A l'aide d'un serveur, ou d'un socket serveur, envoyer un message de rejet de pièces.}
    {L'indication du nombre de pièces rejetés doit s'incrémenter de la valeur envoyé dans le méssage, ainsi que la barre de progression indiquant l'atteinte du seuil de pièces rejeté.\\
    Remarque : Le client n'est pas responsable d'avertir l'utilisateur du dépassement de ce seuil. Il s'agit d'un message d'erreur envoyé par le serveur.}

\Test{Client Windows} % TODO : est ce le bon comportement ?
{Reception d'un avertissement (warning)}
    {A l'aide d'un serveur, ou d'un socket serveur, envoyer un avertissement.}
    {L'avertissement doit s'inscricre dans la liste des messages de log.}

\Test{Client Windows}
{Reception d'un message d'erreur}
	{A l'aide d'un serveur, ou d'un socket serveur, envoyer un message d'erreur.}
	{Une boite de dialogue modale doit apparaitre permettant à l'utilisateur de de faire le choix de continuer ou de stopper la production.}

\Test{Client Windows}
{Envoi de la commande Launch}
	{CLiquer sur le bouton Launch}
	{Un serveur ou une socket serveur doit recevoir un message Launch.
    Le bouton Launch doit se griser, tandis que le bouton Stop doit se dégriser.}

\Test{Client Windows}
{Envoi de la commande Resume}
	{Cliquer sur le bouton Resume}
	{Un serveur ou une socket serveur doit recevoir un message Resume.
    Le bouton Resume doit se griser tandis que le bouton Stop doit se dégriser.}

\Test{Client Windows}
{Envoi de la commande Resume}
	{Cliquer sur le bouton Stop}
	{Un serveur ou une socket serveur doit recevoir un message Stop.
    Le bouton Stop doit se griser tandis que les bouton Launch et Resume doivent se dégriser.}

\Test{Client Windows}
{Envoi d'une configuration}
	{Remplir les champs de configuration}
	{Un serveur ou une socket serveur doit recevoir un message de configuration comportant les bonnes valeurs des champs de configuration.
    Le seuil d'acceptation de pièces rejetés doit s'actualiser avec la valeur envoyé dans le message de configuration.    
    \\Remarque : si aucune valeur n'est rentré, le comportement est incertain.}

    

