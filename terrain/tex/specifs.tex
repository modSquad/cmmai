\chapter{Complément de spécifications}

Ce chapitre décrit les éléments de spécifications qui ne sont pas indiqués dans
le cahier des charges. Ce complément à pour objectif de préciser le
comportement de l'application et a donc une influence directe sur la
réalisation du projet.

Le serveur sera hébergé sur le poste \textit{VxWorks}.

\section{Impression des étiquettes}

\begin{itemize}
	\item Les imprimantes sont sollicitées en tour à tour, ceci permet
d'équilibrer la charge sur les imprimantes et de répartir l'usure.
	\item L'état de fonctionnement de l'imprimante est testé lors de
l'impression d'une étiquette. Si l'impression s'est révélée impossible, alors
la seconde imprimante sera sollicitée.
\end{itemize}

\section{Journaux d'exploitatation}

Les éléments enregistrés dans les journaux d'exploitation sont les suivants :

\begin{itemize}
	\item 
\end{itemize}

L'enregistrement d'un message dans les journaux d'exploitation s'effectue
suffisament rapidement pour considérer la date d'enregistrement du message
comme la date d'un événement (imprecision inférieuse à une seconde, tandis que
l'horloge est précise à la seconde).

\section{Client windows}

Le superviseur, qui accède à l'application à partir du logiciel client sous
windows doit indiquer les paramètres de configuration suivants :

Au lancement de l'application :\\
\begin{itemize}
	\item Le seuil de pièces deffectueuses acceptées
\end{itemize}

Pour chaque nouveau lot à traiter :\\
\begin{itemize}
	\item Le code opérateur (permet d'identifier le superviseur),
	\item la référence du lot (permet d'identifier le lot traité),
	\item le nombre de cartons du lot,
	\item le nombre de pièces par carton.
\end{itemize}

\section{Liste des erreurs et anomalies émises par le serveur}

\begin{itermize}
	\item Le seuil des pièces deffectueuses est atteint,
	\item La file de cartons est pleine,
	\item Il n'y a plus de cartons disponible,
	\item Il n'y a plus de pièce disponible,
	\item Une imprimante est en panne (\textit{anomalie}).
\end{itemize}

\section{Traitement des erreurs}

Le processus de gestion des erreurs est le suivant :

\begin{enumerate}
	\item Fermeture du clapet
	\item Allumer le voyant rouge de l'atelier
	\item Génération d'un message destiné au superviseur
	\item Écriture du message dans les journaux d'exploitation
	\item Passage en mode interactif (mode dialogue) côté client : demande de
reprise d'activité ou terminaison de l'application. 
\end{enumerate}

\section{Traitement des anomalies}

\begin{enumerate}
	\item Allumer le voyant orange de l'atelier
	\item Génération d'un message destiné au superviseur
	\item Écriture du message dans les journaux d'exploitation
\end{enumerate}

\section{Transactions sur le réseau}
Le protocole retenu repose sur des messages de type texte, afin de s'abstraire
des problèmatiques d'architecture materielle entre les différents systèmes : le
serveur sous \textit{VxWorks} (architecture \textit{16 bits}) et le client sous
Windows (architecture \textit{intel 32 ou 64 bits}).

Liste des types de messages du protocole :\\

\begin{itemize}
	\item Client vers serveur
	\begin{description}
		\item[Configuration] décrit la configuration du traitement d'un lot,
		\item[Launch] indique le lancement du traitement d'un lot, 
		\item[Stop] demande la fin de l'application à la fin du traitement du
		lot courrant,
		\item[Resume] demande la reprise de l'application (après une erreur).
	\end{description}

	\item Serveur vers client
	\begin{description}
		\item[Log] envoie un message du journal d'exploitation,
		\item[Warnings] envoie un code d'alerte décrivant une anomalie,
		\item[Error] envoie un code décrivant une erreur,
		\item[Accepted] envoie le nombre de pièces acceptées dans le lot,
		\item[Rejected] envoie le nombre de pièces rejetées dans le lot.
	\end{description}
\end{itemize}
