\chapter{Notice d'intégration}

Ce chapitre présente les différents éléments constituant l'application qui sont
partagés entre les lots de travail, les conventions à respecter lors de la
réalisation du projet afin de faciliter l'intégration et l'organisation de
l'étape d'intégration.

\section{Découpage en lots}

\begin{itemize}
	\item \textbf{Lot n$\degree{}$1} : Communication (serveur sockets, journal
d'événements) (Maxime, paul),
	\item \textbf{Lot n$\degree{}$2} : Gestion du conditionnement et de l'impression
(Yoann, Monica),
	\item \textbf{Lot n$\degree{}$3} : Tâche mère, IHM (client windows), gestion de
l'arrêt d'urgence (Martin, Etienne).
\end{itemize}

\section{Ressources partagées}

Nous avons pris la décision de donner la responsabilité des ressources
partagées entre différentes tâches à la tâche mère $masterTask$.

Les ressources partagées sont les suivantes :

\begin{itemize}
	\item $endSync$ sémaphore de synchronisation notifiant la fin de
l'application,
	\item $boxHandlingRequest$ sémaphore de synchronisation indiquant à la
tache $boxManager$ qu'elle doit activer le traitement du conditionnement,
	\item $boxesQueue$ file d'attente des cartons pleins prêt à être étiquetés
par la tache $printManager$,
	\item $eventsQueue$ file d'attente des événements survenus lors de
l'exploitation,
	\item $logsEventQueue$ file d'attente des messages à écrire dans le journal
des événements,
	\item $socket$ socket réseau permettant la communication entre le client
(opérateur manipulant l'application windows) et le serveur sur l'application en
temps réel.
\end{itemize}

\section{Types et données partagées}

Les données, structures et type de données partagées sont regroupés dans un
fichier d'en-tête générique $boxingServer.h$, en voici les principaux :

\begin{itemize}
	\item les structures de données manipulées par les files d'attente
($event_msg_t$ et $boxesQueueMsg_t$),
	\item les structures des mémoires partagées ($boxData_t$ et $settings_t$),
	\item les événements survenant lors de l'exploitation sont représentés par
une valeur constante, dont la valeur numérique peut être définie manuellement
lorsque l'événement est transmis sur le réseau à destination du client Windows.
\end{itemize}

\section{Guide de style}

Nous avons mis en place des recommandations de développement pour nous assurer
que le code produit par les trois équipe soit cohérent et que les conflits
seront limités. Ainsi, un guide de style a été rédigé pour mettre l'accent sur
la nomenclature des symbôles utilisés dans le code, l'encapsulation des
opérations internes des tâches, et le respect de conventions facilitant les
échanges entre les taches.

\section{Interfaces}

Nous avons également mis en place des interface permettant d'abstraire la
manipulation des périphériques (capteurs, imprimantes, clapets) dans l'optique
de simplifier les tests et la simulation de l'application. On supposera que
pour passer l'application en production, il nous suffira d'utiliser une
implémentation de ces interfaces contrôlant les périphériques, en utilisant,
par exemple, des règles de compilation différentes dans le $Makefile$ de
l'application.

\section{Processus d'intégration}

Le processus d'intégration a lui-même été découpé en plusieurs \textit{lots} de
manière à répartir le travail. La mise en place en amont du guide de style et
des conventions à respecter simplifie grandement l'opération, qui devient
essentiellement une tâche de câblage (réaliser les liaisons entre les
différentes tâches) et de correction du code.

Ainsi, les opérations suivantes ont pu être traitées simultanément :

\begin{itemize}
	\item Rédaction du plan de test d'intégration et élaboration des tests,
	\item câblage des modules,
	\item construction du simulateur,
	\item préparation du déploiement de l'application.
\end{itemize}

\section{Déploiement}

Le déploiement de l'application à nécessité la configuration de l'environnement
de travail \textit{Windriver Workbench} pour utiliser correctement le
simulateur \textit{VxWorks}.

Il faut installer une carte réseau virtuelle \kw{WRTAP}, pour avoir la possibilité de communiquer entre l'hôte et
la cible (à savoir la machine virtuelle VxWorks). 
On doit donner à la carte réseau virtuelle l'adresse \kw{192.168.200.254}.
Il faut ensuite installer le service \kw{vxsimnetd}. Pour cela, il s'agit
d'utiliser l'exécutable \kw{vxsimnetds\_inst}, qui se trouve dans le répertoire
d'installation de Windriver Workbench.

Ensuite, pour configurer le simulateur, placer l'IDE Workbench dans la vue
Remote Systems (en haut à droite de la fenêtre). Sélectionner le simulateur
\kw{vxsirn0}. Dans le menu contextuel qui est proposé lors d'un clic droit sur
cette machine virtuelle, choisir l'option \emph{propriété}. Dans le premier
onglet, choisir \kw{Advanced Boot Parameters}. Sélectionner \kw{simnet} dans la
liste déroulante \kw{Boot device}. Fixer l'adresse IP de la machine virtuelle à
\kw{192.168.200.1}. Valider et lancer le simulateur, qui dispose alors de l'IP
locale ainsi définie.

Il s'agit alors de tester la configuration mise en place. La commande \kw{ping
192.168.200.1} doit répondre, sur une intallation de Windows standard,
quatre lignes, avec des temps très faible, puisque le roundtrip mesuré est effectué
sur la boucle locale.

Le client Windows a été réalisé à l'aide du Framework Qt pour C++. Nous avons
donc choisi une stratégie de déploiement adaptée à des postes n'ayant pas les pré-requis (bibliothèques Qt) disponibles.

\subsection{Installation de la bibliothèque Qt}

Qt est une bibliothèque multi-plateforme, disponible pour Windows, MacOS,
GNU/Linux, proposant, entre autre, un ensemble de composants graphiques et réseaux. Notre
application utilise cette bibliothèque, afin de simplifier et de rendre portable
les parties réseau et interface graphique. Le guide ci-dessous permet d'installer, à partir d'une
installation vierge de Windows (toute version), la bibliothèque Qt, son
environnement de développement, et l'application permettant de contrôler le
système à distance.

\begin{itemize}
    \item Télécharger Qt pour Windows à l'adresse :\\ \kw{http://qt.nokia.com/downloads/sdk-windows-cpp}.
    \item Double-cliquer sur le fichier téléchargé, et suivre les étapes
    d'installation (pour une machine x86, sur un Windows XP, Vista ou Seven,
	    cette procédure consiste à cliquer sur suivant à chaque écran, et à
	    accepter la licence LGPL proposée).
    \item Si l'on désire compiler le programme à partir des sources, ouvrir le
    fichier d'extension \kw{.pro} dans QtCreator, l'IDE de Qt, et cliquer sur
    la flèche verte en bas à droite de l'écran. Cela compile et lance le
    programme. Ceci est la méthode recommandée pour utiliser ou tester le
    programme de contrôle, puisqu'il installe de fait correctement
    l'environnement Qt.
    \item Si l'on désire exécuter le programme sans l'avoir compilé
    préalablement, il faut s'assurer qu'il a été compilé pour la bonne
    architecture, et pour le bon type de système. Il s'agit, comme un programme
    Windows classique, de double cliquer sur l'exécutable généré par la
    compilation. Il est impératif que les bibliothèques Qt soient installées
    dans le système, et soient dans le chemin de recherche des bibliothèques
    partagées. De nombreuses entreprises d'informatique embarquée
    utilisent Qt, ce point ne devrait donc pas trop poser de problème.
\end{itemize}

